\section{Self-Organizing Autoencoder (SOAE)}\label{sec:soae}

First we recall what a SOM is.
Then we extend the formulation to the deep learning context.
At last,
we compare the deep version of SOM with VAE\@.

\subsection{Self-Organizing Map}

Self-organizing maps often comes with two formulations:
one is the original incremental algorithm;
the other is a batch version.
We use the original incremental version.

First define a set of latent nodes
\( \cZ=\{z_1,\dots,z_M\} \)
on a two-dimensional plane.
Typically,
points in \( \cZ \) are chosen to form a rectangular or hexagonal lattice.
The distances between latent nodes are used to define neighborhoods
through a \emph{neighborhood function} \( h_t(z',z) \).
Each latent node \( z_i \) is related to data space by a \emph{reference vector}
\( m_i\in\cX \).
Reference vectors are initialized first,
and then refined by the Algorithm~\ref{alg:som}.

\begin{algorithm}[tb]
    \caption{SOM~\cite{kohonen1982self, kohonen1998self}}\label{alg:som}
 \begin{algorithmic}
    \STATE {\bfseries Input:}
        latent nodes \(\cZ\subset\real^2\),
        data space \(\cX\subset\real^n\),
        learning rate \(\alpha_t\),
        neighborhood function \(h_t(z',z)\),
        max iteration time \(T\)
    \STATE
    \STATE Initialize reference vectors \(m_i\in\real^n\) for each \(z_i\in\cZ\)
    \FOR{\(t=1\) {\bfseries to} \(T\)}
    \STATE Sample \(x\in\cX\)
    \STATE Find \emph{best matching node} \(m_c\),
        where
        \begin{equation}
            c = \argmin_i \big\{\|x-m_i\|\big\}
        \end{equation}
    \STATE Update reference vectors
        \begin{equation}\label{eq:som_update}
            m_i \leftarrow m_i + \alpha_t h_t(z_c,z_i)\big(x-m_i\big),\quad\forall i
        \end{equation}
    \ENDFOR
\end{algorithmic}
\end{algorithm}

The neighborhood kernel can use the Gaussian function,
\begin{equation}
    h_t(z',z) = e^{-\frac{\|z' - z\|^2}{2\sigma^2(t)}},
\end{equation}
where \(t\) is iteration time,
and \(\sigma(t)\) defines the width of the neighborhood kernel.
Both the learning rate \(\alpha_t\) and \(\sigma(t)\) are monotonically decreasing functions of time.

Note that the update rule~\eqref{eq:som_update} is a step to optimize the loss
\begin{equation}
    \min_{m_i} \frac{1}{2}h_t(z_c,z_i)\|x-m_i\|^2.
\end{equation}

\subsection{Deep Version}

The original SOM is a shallow neural network.
We propose a deep version.
Changes are,

\begin{itemize}
    \item Discrete two-dimensional latent nodes become a continuous latent space \(\cZ\subset\real^m\) with dimension \(m\).
    \item Difference between two points in data space \(\cX\) can be measured by any given distance function \(d(\cdot,\cdot)\),
        \eg, the distance induced by the L1/L2 norm.
    \item Reference vector of a latent vector \(z\in\cZ\) is represented by a deep network \(G\), \ie, \(G(z)\in\real^n\).
    \item The best matching node is represented by another network \(C\).
        Given a sample \(x\in\cX\),
        the best matching node \(C(x)\) is found via minimizing the loss,
        \begin{equation}\label{eq:soae_c}
            \min_C d(x, G(C(x))).
        \end{equation}
        Here the network \(G\) is fixed.
    \item For any given \(z'\in\cZ\) and iteration \(t\),
        the integration of the neighborhood function \(h_t(z',z)\) over the latent space \(\cZ\) is required to be \(1\),
        and we call it the \emph{neighborhood kernel}.
        \begin{equation}
            \int_\cZ h_t(z',z)dz = 1.
        \end{equation}
    \item At step \(t\), the update rule~\eqref{eq:som_update} now becomes a single optimization step of \(G\) for the objective,
        \begin{equation}\label{eq:deep_update}
            \min_G \int_{\cZ}h_t(C(x), z)d(x, G(z))dz.
        \end{equation}
\end{itemize}

The update step in~\eqref{eq:deep_update} contains an integration over the latent space.
For given \(C\), \(x\), and \(t\),
treat \(h_t(C(x), \cdot)\) as a probability distribution over \(\cZ\).
Rewrite the integration as an expectation,
we have
\begin{equation}\label{eq:soae_update}
    \min_G \Ebb_{z\sim h_t(C(x), \cdot)}\big[d(x, G(z))\big].
\end{equation}
Then we can optimize this approximately by sampling some finite number of latent vectors.
For example,
in the case of sampling only one point \(z\),
the objective is,
\begin{equation}\label{eq:soae_update_one}
    \min_G d(x, G(z)).
\end{equation}
We summarize the procedure in Algorithm~\ref{alg:soae}.

\begin{algorithm}[tb]
    \caption{SOAE}\label{alg:soae}
 \begin{algorithmic}
    \STATE {\bfseries Input:}
        latent space \(\cZ\subset\real^m\),
        data space \(\cX\subset\real^n\),
        neighborhood kernel \(h_t(z',z)\),
        max iteration time \(T\),
        \(C\) iteration time \(T_1\),
        \(G\) iteration time \(T_2\)
    \STATE
    \STATE Initialize network \(G\), \(C\)
    \FOR{\(t=1\) {\bfseries to} \(T\)}
    \FOR{\(t_1=1\) {\bfseries to} \(T_1\)}
        \STATE Sample \(x\in\cX\)
        \STATE Update \(C\) with loss~\eqref{eq:soae_c}.
    \ENDFOR
    \FOR{\(t_2=1\) {\bfseries to} \(T_2\)}
        \STATE Sample \(z\sim h_t(C(x), \cdot)\)
        \STATE Update \(G\) with loss~\eqref{eq:soae_update_one}.
    \ENDFOR
    \ENDFOR
\end{algorithmic}
\end{algorithm}

\subsection{Relation to Autoencoder}

In the context of autoencoder,
\(G\) is the decoder,
and \(C\) is the encoder.
Equation~\eqref{eq:soae_c} is like the autoencoder loss.
When training \(G\),
the sampling of \(z\sim h_t(C(x), \cdot)\) is like the variational autoencoder~\cite{kingma2013auto}.
However,
we no not need the reparametrization trick because the gradient of \(z\) need not to propagate to network \(C\).
The kl-divergence loss counterpart for regularizing conditional distribution \(P(z|x)\) is absent from loss here,
because the neighborhood kernel took its place.
It also resembles denoising autoencoder~\cite{vincent2008extracting},
albeit denoising the latent space rather than the data space.
The training is like wake-sleep algorithm~\cite{hinton1995wake}.

In summary,
from the algorithmic view,
SOAE is like training an encoder \(C\) and decoder \(G\) using the wake-sleep algorithm.
During the wake pass, \ie, training of \(G\),
it resembles a VAE\@;
while in the sleep pass, \ie, training of \(C\),
it is like a normal autoencoder.
