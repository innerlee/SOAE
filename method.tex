\section{SOAE}\label{sec:soae}

First we recall what a SOM is.
Then we extend the formulation to the deep learning context.
At last,
we compare the deep version of SOM with VAE\@.

\subsection{Self-Organizing Map}

Self-organizing maps often comes with two formulations:
one is the original incremental algorithm;
the other is a batch version.
We use the original incremental version.

First define a set of latent nodes
\( \cZ=\{z_1,\dots,z_M\} \)
on a two-dimensional plane.
Typically,
points in \( \cZ \) are chosen to form a rectangular or hexagonal lattice.
The distances between latent nodes are used to define neighborhoods
through a \emph{neighborhood function} \( h_{z'}(z) \).
Each latent node \( z_i \) is related to data space by a \emph{reference vector}
\( m_i\in\cX \).
Reference vectors can be initialized randomly,
and refined by the algorithm.
